\documentclass[12pt,twoside,a4paper]{article}
\usepackage{color}


\begin{document}
\title{\color{blue}The Building Blocks and Tools JavaScript Library.}
\author{F. Quinonez$^{1,2,3}$ and A. J. Hern\'{a}ndez G\'{o}ez$^{1}$\\
  \small{\em{1. CLEVER STATISTICS SAS, Barrancabermeja Colombia.}}\\
\small{\em{2. Unidades Tecnol\'ogicas de Santander, Departamento de Ciencias B\'asicas.}}\\
\small{\em{3. Universidad Industrial de Santander, Escuela de F\'isica.}}
}
\maketitle
\begin{abstract}
The Building Blocks and Tools library (or simply the BBT library) is a JavaScript
library created to offer to its users a web based framework for generating, visualizing and analysing data across the web. The BBT library is
inspired by the three software most used in High Energy Physics and Biomedical Physics research, that are: ROOT, GEANT4 and GATE;
The BBT library rest on another open source JavaScript libraries like D3, THREE, MathJax, FileSaver
and Requirejs. BBT exploits the power of the modern web browsers
and does not require installation as its parents built with C++ do, so
the users can start their analysis as soon as they download the library, thought 5 minutes against the approximated 3 hours that takes build 
ROOT, GEANT4 AND GATE.
The BBT library due to be a web technology have a great potential to be a
massive tool for any kind of users that want to work (or play) with data, to
boost learning and teaching Big-Data, to boost initiatives of Citizen
Science, incentivating Web/Computing Based Discovery WCBD and initiatives aiming that the great Research Collaborations share
their data generated and produced in these experiments. The BBT library also wants to become the backbone of Mobile Applications dealing with data. 
The BBT library Allow visualization in real time, not as GRID or batch systems like GISELA,
lxplus, etc. etc. wherein users can not see visualization in real time because the visualization drivers are in the machines that are running the jobs.
\end{abstract}

\end{document}
